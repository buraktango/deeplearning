\documentclass{article}
\usepackage{enumitem,graphicx,fancyhdr,amsmath,amssymb,amsthm,amsfonts,subfig,url,hyperref}
\usepackage[margin=1in]{geometry}

%----------------------- Macros and Definitions --------------------------

\author{}
\title{}
\date{\vspace{-5ex}}

\newcommand{\suid}{mathieu.boudreau2}
\newcommand{\chapnumber}{2}

\renewcommand{\theenumi}{\bf \Alph{enumi}}

\fancypagestyle{plain}{}
\pagestyle{fancy}
\fancyhf{}
\fancyhead[RO,LE]{\sffamily\bfseries\large Mathieu Boudreau}
\fancyhead[LO,RE]{\sffamily\bfseries\large Supp. Mat. for Deep Learning Book}
\fancyfoot[LO,RE]{\sffamily\bfseries\large \suid @mail.mcgill.ca}
\fancyfoot[RO,LE]{\sffamily\bfseries\thepage}
\renewcommand{\headrulewidth}{1pt}
\renewcommand{\footrulewidth}{1pt}

\graphicspath{{figures/}}

\usepackage{bm}

% Matrices
\def\mA{{\bm{A}}}
\def\mB{{\bm{B}}}
\def\mI{{\bm{I}}}


%-------------------------------- Title ----------------------------------

\title{Chapter $\chapnumber$: Exercise Set}

%--------------------------------- Text ----------------------------------

\begin{document}
\maketitle

\section*{Exercise \chapnumber.1}
Consider the following matrices,

\begin{equation}
\mA
=
\begin{bmatrix}
    1 & 3 & 5 \\
    2 & 4 & 6 \\
    0 & 8 & 2 
\end{bmatrix}
\end{equation}

\begin{equation}
\mB
=
\begin{bmatrix}
    7 & 2 \\
    1 & 5 \\
    9 & 4 
\end{bmatrix}
\end{equation}

Calculate the following values/matrices:

\begin{enumerate}[label=(\alph*)]

\item $\textit{A}_{2,3}$
\item $\mA^T$
\item $\mB^T$
\item $\mA+\mA$
\item $2\mB+1$
\item $\mA\mA$
\item $\mA\mB$
\item $\mA\odot\mA$
\item $(\mI_{3}\mB)\mI_{2}$

\end{enumerate} 

\section*{Exercise \chapnumber.2}
Write the following set of equations into the matrix form $\mA \vx = \vb$.

\begin{equation}
\begin{array}{rcr} 
2x_{1} + 3x_{2} + x_{3} + 8x_{4} & = & 5 \\
x_{1} - x_{2} + x_{3} - x_{4} & = & 2 \\
4x_{1} + 5x_{3} - 2x_{4}& = & -4 \\
6x_{1} - 5x_{2} + 3x_{3} - 9x_{4} & = & 0
\end{array}
\end{equation}

\section*{Exercise \chapnumber.3}
Let $\sV$ be the set of vectors \{$\vv^{(1)}, \vv^{(2)}$\},

\begin{equation}
\vv^{(1)}
=
\begin{bmatrix}
    1\\
    0
\end{bmatrix},\ 
\vv^{(2)}
=
\begin{bmatrix}
    0\\
    1
\end{bmatrix}
\end{equation}

Find the values of the coefficients $c_{i}$ such that:

\begin{equation}
\begin{bmatrix}
    1/2\\
    4
\end{bmatrix} 
=
\sum\limits_{i}c_{i}\vv^{(i)}
\end{equation} 

\section*{Exercise \chapnumber.4}
Let $\sV$ be the set of vectors \{$\vv^{(1)}, \vv^{(2)}$\},

\begin{equation}
\vv^{(1)}
=
\begin{bmatrix}
    2\\
    -1
\end{bmatrix},\ 
\vv^{(2)}
=
\begin{bmatrix}
    -4\\
    2
\end{bmatrix}
\end{equation}

\begin{enumerate}[label=(\alph*)]

\item Which of the following vectors are in the span of $\sV$?

\begin{equation}
\begin{bmatrix}
    2\\
    0
\end{bmatrix} 
,
\begin{bmatrix}
    -10\\
    -5
\end{bmatrix} 
,
\begin{bmatrix}
    -10\\
    5
\end{bmatrix}
,
\begin{bmatrix}
    0\\
    0
\end{bmatrix}  
\end{equation} 

\item Are the vectors in the set $\sV$ linearly independent?
\end{enumerate}

\section*{Exercise \chapnumber.5}
Consider the matrices:

\begin{equation}
\mA
=
\begin{bmatrix}
    1 & 0 & 0 \\
    0 & 1 & 1 \\
    1 & 0 & 1 
\end{bmatrix}
,
\vb
=
\begin{bmatrix}
    42 \\
    0 \\
    12 
\end{bmatrix}
\end{equation} 

\begin{enumerate}[label=(\alph*)]

\item Is $\vb$ in the range of $\mA$?

\item If so, solve $\mA \vx = \vb$ for $\vx$.  If not, describe why.
\end{enumerate}

\section*{Exercise \chapnumber.6}
Consider the matrices:

\begin{equation}
\mA
=
\begin{bmatrix}
    1 & 2 & 2 \\
    4 & 5 & 3 \\
    4 & 5 & 3
\end{bmatrix}
,
\vb
=
\begin{bmatrix}
    42 \\
    0 \\
    12
\end{bmatrix}
\end{equation} 

\begin{enumerate}[label=(\alph*)]

\item Is $\vb$ in the range of $\mA$?

\item If so, solve $\mA \vx = \vb$ for $\vx$.  If not, describe why.
\end{enumerate}

\section*{Exercise \chapnumber.7}

Consider the two vectors

\begin{equation}
\vv_{1}
=
\begin{bmatrix}
    5\\
    -8\\
    1\\
    100\\
    4\\
    7
\end{bmatrix},\ 
\vv_{2}
=
\begin{bmatrix}
    0\\
    0\\
    0\\
    100\\
    0\\
    0
\end{bmatrix}
\end{equation}

\begin{enumerate}[label=(\alph*)]

\item Calculate the $\textit{L}^{1}$ norm of $\vv_{1}$ and $\vv_{2}$.

\item Calculate the $\textit{L}^{2}$ norm of $\vv_{1}$ and $\vv_{2}$.
\end{enumerate}

\section*{Exercise \chapnumber.8}

Consider the two vectors

\begin{equation}
\vv_{1}
=
\begin{bmatrix}
    5\\
    0\\
\end{bmatrix},\ 
\vv_{2}
=
\begin{bmatrix}
    3\\
    4
\end{bmatrix}
\end{equation}

\begin{enumerate}[label=(\alph*)]

\item Calculate the $\textit{L}^{1}$ norm of $\vv_{1}$ and $\vv_{2}$.

\item Calculate the $\textit{L}^{2}$ norm of $\vv_{1}$ and $\vv_{2}$.
\end{enumerate}

\section*{Exercise \chapnumber.9}
Imagine you are developping an image compression algorithm. You expect that after a certain image transformation (e.g. Wavelet), the resulting matrix can be well aproximated by a sparse matrix. Which type of norm ($\textit{L}^{1}$ vs. $\textit{L}^{2}$) do you think will be the best choice to quantify sparsity in your algorithm? Explain.

\section*{Exercise \chapnumber.10}

Which type of norm ($\textit{L}^{1}$ vs. $\textit{L}^{2}$) would provide the best estimate of the minimal distance needed for:

\begin{enumerate}[label=(\alph*)]
\item a boat to sail to a nearby harbour?

\item a New York taxicab to drive the customer to her/his destination?
\end{enumerate}

\section*{Exercise \chapnumber.11}

Consider the following matrix,

\begin{equation}
\mA
=
\begin{bmatrix}
    11 & -3\\
    18 & -4
\end{bmatrix}
\end{equation}

Which of the following vectors are eigenvectors of $\mA$? What are their eigenvalues?

\begin{enumerate}[label=(\alph*)]

\item
\begin{equation}
\begin{bmatrix}
    1 & -1
\end{bmatrix}^T
\end{equation}

\item
\begin{equation}
\begin{bmatrix}
    1 & 3
\end{bmatrix}^T
\end{equation}

\item
\begin{equation}
\begin{bmatrix}
    -1 & 3
\end{bmatrix}^T
\end{equation}

\item
\begin{equation}
\begin{bmatrix}
    2 & 4
\end{bmatrix}^T
\end{equation}


\end{enumerate}

\end{document}
